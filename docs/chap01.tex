%%
% 引言或背景
% 引言是论文正文的开端,应包括毕业论文选题的背景、目的和意义;对国内外研究现状和相关领域中已有的研究成果的简要评述;介绍本项研究工作研究设想、研究方法或实验设计、理论依据或实验基础;涉及范围和预期结果等。要求言简意赅,注意不要与摘要雷同或成为摘要的注解。
%%
% 章、节、小节、图片、公式、表格下方的\label{...}标记不建议删除,因为这些可以做到自动引用的作用,当某些公式、图片被删除时,\label{...}标记能使正文中的编号自动更新,省去一个一个编号的麻烦。

\chapter{绪论}
\label{cha:introduction}
\section{引言}
\label{sec:prologue}
引言是论文正文的开端,应包括毕业论文选题的背景、目的和意义;对国内外研究现状和相关领域中已有的研究成果的简要评述;介绍本项研究工作研究设想、研究方法或实验设计、理论依据或实验基础;涉及范围和预期结果等。要求言简意赅,注意不要与摘要雷同或成为摘要的注解。

\section{国内外研究现状和相关工作}
\label{sec:related_work}
对国内外研究现状和相关领域中已有的研究成果的简要评述。

\section{本文的论文结构与章节安排}
\label{sec:arrangement}
本文共分为五章,各章节内容安排如下:

第一章为绪论。

第二章为本模板遵循的排版及格式。

第三章为图像的插入示例。

第四章为公式与表格的插入示例

第五章是本文的最后一章,结论与展望。是对本文内容的整体性总结以及对未来工作的展望。

