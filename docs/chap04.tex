\chapter{公式与表格的插入示例}
\label{cha:for_tab_example}
公式用于对论文基础理论的介绍,表格则是对一些不方便进行作图的数据进行展示。

\section{公式的插入}
\label{sec:formula}
带左半边大括号的核反应方程式,如式(\ref{eqn:fusion_reactions})所示:
\begin{equation}
	\label{eqn:fusion_reactions}
	\left\{
	\begin{aligned}
		&\mbox{D}+\mbox{D}\rightarrow \mbox{T}\,(\text{1.01}\;\mbox{MeV})+\mbox{p}\,(\text{3.03}\;\mbox{MeV}) \\
		&\mbox{D}+\mbox{D}\rightarrow {^{\text{3}}}\mbox{He}\,(\text{0.82}\;\mbox{MeV})+\mbox{n}\,(\text{2.45}\;\mbox{MeV}) \\
		&\mbox{D}+\mbox{T}\rightarrow \text{α}\,(\text{3.52}\;\mbox{MeV})+\mbox{n}\,(\text{14.06}\;\mbox{MeV}) \\
		&\mbox{D}+{^{\text{3}}}\mbox{He}\rightarrow \text{α}\,(\text{3.67}\;\mbox{MeV})+\mbox{p}\,(\text{14.67}\;\mbox{MeV})
	\end{aligned}
	\right.
\end{equation}

狄拉克函数$\delta_{ij}$的表达式:
\begin{equation}
	\label{eqn:delta_ij}
	\delta_{ij}=\left\{
	\begin{aligned}
		1&    &\mbox{if}&    &i=j \\
		0&    &\mbox{if}&    &i\neq j
	\end{aligned}
	\right.
\end{equation}

一般的公式:
\begin{equation}
	\label{eqn:vec_v_cm}
	\vec{v}_{cm}=\dfrac{m_{1}\vec{v}_{1}+m_{2}\vec{v}_{2}}{m_{1}+m_{2}}
\end{equation}

超长的公式:
\begin{equation}
	\label{eqn:iint_theta3_phi3}
	\begin{split}
		\int_{0}^{\pi}\int_{0}^{2\pi} \sin\theta_{3}&\dfrac{\exp(-\alpha v_{cm}^{2})}{v_{cm}}\sinh(\mu \gamma v_{r}v_{cm})\mbox{d}\phi_{3}\mbox{d}\theta_{3}=\dfrac{2\pi \sqrt{\pi}}{4\sqrt{\alpha}v_{3}u_{3}}\exp\left( \dfrac{(\mu \gamma v_{r})^{2}}{4\alpha} \right) \\
		&\times \Bigg( \mbox{erf}\left( \dfrac{\mu \gamma v_{r}+2\alpha(v_{3}-u_{3})}{2\sqrt{\alpha}} \right)-\mbox{erf}\left( \dfrac{-\mu \gamma v_{r}+2\alpha(v_{3}-u_{3})}{2\sqrt{\alpha}} \right) \\
		&+\mbox{erf}\left( \dfrac{-\mu \gamma v_{r}+2\alpha(v_{3}+u_{3})}{2\sqrt{\alpha}} \right)-\mbox{erf}\left( \dfrac{\mu \gamma v_{r}+2\alpha(v_{3}+u_{3})}{2\sqrt{\alpha}} \right) \Bigg)
	\end{split}
\end{equation}

输入矩阵:
\begin{equation}
	\label{eqn:matrix}
	\textbf{H} = \begin{bmatrix}
		I*\mybold{x}_i \\ \textbf{h}
	\end{bmatrix}
\end{equation}