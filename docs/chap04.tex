\chapter{公式、表格与代码的插入示例}
\label{cha:for_tab_example}
公式用于对论文基础理论的介绍,表格则是对一些不方便进行作图的数据进行展示。

\section{公式的插入}
\label{sec:formula}
带左半边大括号的核反应方程式,如式(\ref{eqn:fusion_reactions})所示:
\begin{equation}
	\label{eqn:fusion_reactions}
	\left\{
	\begin{aligned}
		&\mbox{D}+\mbox{D}\rightarrow \mbox{T}\,(\text{1.01}\;\mbox{MeV})+\mbox{p}\,(\text{3.03}\;\mbox{MeV}) \\
		&\mbox{D}+\mbox{D}\rightarrow {^{\text{3}}}\mbox{He}\,(\text{0.82}\;\mbox{MeV})+\mbox{n}\,(\text{2.45}\;\mbox{MeV}) \\
		&\mbox{D}+\mbox{T}\rightarrow \text{α}\,(\text{3.52}\;\mbox{MeV})+\mbox{n}\,(\text{14.06}\;\mbox{MeV}) \\
		&\mbox{D}+{^{\text{3}}}\mbox{He}\rightarrow \text{α}\,(\text{3.67}\;\mbox{MeV})+\mbox{p}\,(\text{14.67}\;\mbox{MeV})
	\end{aligned}
	\right.
\end{equation}

狄拉克函数$\delta_{ij}$的表达式:
\begin{equation}
	\label{eqn:delta_ij}
	\delta_{ij}=\left\{
	\begin{aligned}
		1,\; i=j \\
		0,\; i\neq j
	\end{aligned}
	\right.
\end{equation}

一般的公式:
\begin{equation}
	\label{eqn:vec_v_cm}
	\vec{v}_{cm}=\dfrac{m_{1}\vec{v}_{1}+m_{2}\vec{v}_{2}}{m_{1}+m_{2}}
\end{equation}

超长的公式\cite{appelbe2011production}:
\begin{equation}
	\label{eqn:iint_theta3_phi3}
	\begin{split}
		\int_{0}^{\pi}\int_{0}^{2\pi} \sin\theta_{3}&\dfrac{\exp(-\alpha v_{cm}^{2})}{v_{cm}}\sinh(\mu \gamma v_{r}v_{cm})\mbox{d}\phi_{3}\mbox{d}\theta_{3}=\dfrac{2\pi \sqrt{\pi}}{4\sqrt{\alpha}v_{3}u_{3}}\exp\left( \dfrac{(\mu \gamma v_{r})^{2}}{4\alpha} \right) \\
		&\times \Bigg( \mbox{erf}\left( \dfrac{\mu \gamma v_{r}+2\alpha(v_{3}-u_{3})}{2\sqrt{\alpha}} \right)-\mbox{erf}\left( \dfrac{-\mu \gamma v_{r}+2\alpha(v_{3}-u_{3})}{2\sqrt{\alpha}} \right) \\
		&+\mbox{erf}\left( \dfrac{-\mu \gamma v_{r}+2\alpha(v_{3}+u_{3})}{2\sqrt{\alpha}} \right)-\mbox{erf}\left( \dfrac{\mu \gamma v_{r}+2\alpha(v_{3}+u_{3})}{2\sqrt{\alpha}} \right) \Bigg)
	\end{split}
\end{equation}

输入矩阵:
\begin{equation}
	\label{eqn:matrix}
	A_{m\times n}=
	\left[ {\begin{array}{cccc}
			a_{11} & a_{12} & \cdots & a_{1n}\\
			a_{21} & a_{22} & \cdots & a_{2n}\\
			\vdots & \vdots & \ddots & \vdots\\
			a_{m1} & a_{m2} & \cdots & a_{mn}\\
	\end{array}}\right]
\end{equation}

\section{表格的插入}
\label{sec:table}
插入一般的表格:
\begin{table}[h] %voc table result
	\centering
		\caption{典型的实验对比表格}	
		\begin{tabular}{*{4}{c}}
			\toprule
	 		Method & Pixel Acc. & Mean Acc. & Mean Iu.\\
			\midrule
			Liu等人\cite{liu2011sift}  & 76.7 & - & -\\
		Tighe等人\cite{tighe2013finding}  & 78.6 & 39.2 & -\\
			FCN-16s\cite{long2015fully} & 85.2 & \textbf{51.7} & 39.5\\
			Deeplab-LargeFOV\cite{chen14semantic} & 85.6 & 51.2 & 39.7\\
			\midrule
			Grid-LSTM5 & \textbf{86.2} & 51.0 & \textbf{41.2}\\
			\bottomrule
		\end{tabular}	
		\label{tab:siftflow}
\end{table}


较为复杂的表格:
\begin{table}[h]
	\renewcommand\arraystretch{1.5}
	\centering
	\caption{较为复杂的表格(涉及单元格的合并与拆分)}
	\begin{tabular}{*{5}{c}}
		\toprule
		区域 & \tabincell{c}{外侧核热功率\\(MW)} & \tabincell{c}{内侧核热功率\\(MW)} & 结构 & \tabincell{c}{结构核热功率\\(MW)} \\
		\midrule
		第一壁涂层 & 20.0 & 13.4 & \multirow{2}{*}{第一壁} & \multirow{2}{*}{151.7} \\
		第一壁结构层 & 70.2 & 48.1 & ~ & ~ \\
		\midrule
		Be-1区 & 37.9 & 26.5 & \multirow{4}{*}{氚增殖区} & \multirow{4}{*}{736.2} \\ 
		Li$ _{\text{4}} $SiO$ _{\text{4}} $-1区 & 126.7 & 86.8 & ~ & ~ \\
		Be-2区 & 133.6 & 94.1 & ~ & ~ \\
		Li$ _{\text{4}} $SiO$ _{\text{4}} $-2区 & 134.4 & 96.2 & ~ & ~ \\
		\bottomrule
	\end{tabular}
	\label{tab:chap05_nucheat_tot}
\end{table}

\section{代码的插入}
\label{sec:code}
本模版支持在论文中插入代码片段,或直接从源码文件进行插入。例如,在论文中插入代码片段的效果为:
\begin{python}
def func():
print("hello world")
with open('./output.txt', 'w') as f:
L = f.readlines()

if __name__ == "__main__":
# this is a comment line
func()
\end{python}
也可在行内插入代码片段,例如:Python中重载加法运算符的函数为\pyinline{__add__},类的标识符为\pyinline{class}。
此外,还可直接插入代码文件,例如插入\texttt{./code/demo.cpp}的效果为:
\lstinputlisting[style=sysucpp]{code/demo.cpp}


\section{算法框/伪代码的插入}
本模版支持在论文中插入算法框/伪代码。例如,在论文中插入算法框的效果为:
\begin{algorithm}[h]
    \caption{示例算法}
    \begin{algorithmic}[1]
        \REQUIRE 正整数$n$
        \ENSURE $1$到$n$的和
        \STATE 初始化$x=0$
        \FOR{$i = 0, 1, \cdots, n-1$}
        \STATE x = x + i
        \ENDFOR
        \RETURN x
    \end{algorithmic}
\end{algorithm}