\chapter{数学}
\label{cha:math}
\section{公式的插入}
\label{sec:formula}
带左半边大括号的核反应方程式,如式(\ref{eqn:fusion_reactions})所示:
\begin{equation}
	\label{eqn:fusion_reactions}
	\left\{
	\begin{aligned}
		&\mbox{D}+\mbox{D}\rightarrow \mbox{T}\,(\text{1.01}\;\mbox{MeV})+\mbox{p}\,(\text{3.03}\;\mbox{MeV}) \\
		&\mbox{D}+\mbox{D}\rightarrow {^{\text{3}}}\mbox{He}\,(\text{0.82}\;\mbox{MeV})+\mbox{n}\,(\text{2.45}\;\mbox{MeV}) \\
		&\mbox{D}+\mbox{T}\rightarrow \text{α}\,(\text{3.52}\;\mbox{MeV})+\mbox{n}\,(\text{14.06}\;\mbox{MeV}) \\
		&\mbox{D}+{^{\text{3}}}\mbox{He}\rightarrow \text{α}\,(\text{3.67}\;\mbox{MeV})+\mbox{p}\,(\text{14.67}\;\mbox{MeV})
	\end{aligned}
	\right.
\end{equation}

狄拉克函数$\delta_{ij}$的表达式:
\begin{equation}
	\label{eqn:delta_ij}
	\delta_{ij}=\left\{
	\begin{aligned}
		1,\; i=j \\
		0,\; i\neq j
	\end{aligned}
	\right.
\end{equation}

一般的公式:
\begin{equation}
	\label{eqn:vec_v_cm}
	\vec{v}_{cm}=\dfrac{m_{1}\vec{v}_{1}+m_{2}\vec{v}_{2}}{m_{1}+m_{2}}
\end{equation}

超长的公式\cite{appelbe2011production}:
\begin{equation}
	\label{eqn:iint_theta3_phi3}
	\begin{split}
		\int_{0}^{\pi}\int_{0}^{2\pi} \sin\theta_{3}&\dfrac{\exp(-\alpha v_{cm}^{2})}{v_{cm}}\sinh(\mu \gamma v_{r}v_{cm})\mbox{d}\phi_{3}\mbox{d}\theta_{3}=\dfrac{2\pi \sqrt{\pi}}{4\sqrt{\alpha}v_{3}u_{3}}\exp\left( \dfrac{(\mu \gamma v_{r})^{2}}{4\alpha} \right) \\
		&\times \Bigg( \mbox{erf}\left( \dfrac{\mu \gamma v_{r}+2\alpha(v_{3}-u_{3})}{2\sqrt{\alpha}} \right)-\mbox{erf}\left( \dfrac{-\mu \gamma v_{r}+2\alpha(v_{3}-u_{3})}{2\sqrt{\alpha}} \right) \\
		&+\mbox{erf}\left( \dfrac{-\mu \gamma v_{r}+2\alpha(v_{3}+u_{3})}{2\sqrt{\alpha}} \right)-\mbox{erf}\left( \dfrac{\mu \gamma v_{r}+2\alpha(v_{3}+u_{3})}{2\sqrt{\alpha}} \right) \Bigg)
	\end{split}
\end{equation}

输入矩阵:
\begin{equation}
	\label{eqn:matrix}
	A_{m\times n}=
	\left[ {\begin{array}{cccc}
			a_{11} & a_{12} & \cdots & a_{1n}\\
			a_{21} & a_{22} & \cdots & a_{2n}\\
			\vdots & \vdots & \ddots & \vdots\\
			a_{m1} & a_{m2} & \cdots & a_{mn}\\
	\end{array}}\right]
\end{equation}
